\markboth{BIBLIOGRAFÍA}{BIBLIOGRAFÍA}
\addcontentsline{toc}{chapter}{Bibliografía}


\begin{thebibliography}{99}

% EXEMPLO DE ARTIGO DE REVISTA
\bibitem{Securing-Docker-Containers-from-Denial-of-Service} J. Chelladhurai, P. R. Chelliah e S. A. Kumar, ``\textit{Securing Docker Containers from Denial of Service (DoS) Attacks}'', {\it 2016 IEEE International Conference on Services Computing (SCC)}, San Francisco, CA, 2016, pp. 856-859.

\bibitem{clairWeb} \textit{Clair Documentation}. Documentación oficial de Clair (\url{https://coreos.com/clair/docs/latest/}). Consultado o 1 de xullo do 2018.

% EXEMPLO DE ARTIGO DE REVISTA
\bibitem{To-Docker-Or-Not-To-Docker} T. Combe, A. Martin e R. Di Pietro, ``\textit{To Docker or Not to Docker: A Security Perspective}'', {\it IEEE Cloud Computing, vol. 3, no. 5}, Set.-Out. 2016, pp. 54-62.

\bibitem{docker-content-trust} \textit{Content trust}. Documentación oficial de Docker (\url{https://docs.docker.com/engine/security/trust/content\_trust/}). Consultado o 1 de xullo do 2018.

% EXEMEPLO DE LIBRO
\bibitem{ScrumPrimer} P. Deemer, G. Benefield, C. Larman e B. Vodde, {\it The Scrum Primer}. 2012. \url{http://scrumprimer.org/}

\bibitem{docker-bench-security} \textit{Docker bech security}. Repositorio GitHub (\url{https://github.com/docker/docker-bench-security}). Consultado o 1 de xullo do 2018.

\bibitem{docker-security} \textit{Docker security}. Documentación oficial de Docker (\url{https://docs.docker.com/engine/security/security/}). Consultado o 1 de xullo do 2018.

\bibitem{SELinuxDocker} \textit{Docker SELinux Security Policy}. \textit{Red Hat Costumer Portal} (\url{https://access.redhat.com/documentation/en-us/red_hat_enterprise_linux_atomic_host/7/html/container_security_guide/docker_selinux_security_policy}). Consultado o 1 de xullo do 2018.

% EXEMEPLO DE LIBRO
\bibitem{introduccionContainerSecurityDocker} \textit{Docker Team}, \textit{White Paper: ``Introduction to Container Security''}, 2015.

% EXEMPLO DE ARTIGO DE REVISTA
\bibitem{dockerQoS} A. Dusia, Y. Yang e M. Taufer., ``\textit{Network Quality of Service in Docker Containers}'', {\it S2015 IEEE International Conference on Cluster Computing, Chicago, IL}, 2015, pp. 527-528.

% EXEMPLO DE ARTIGO DE REVISTA
\bibitem{MSO4SC} C. Fernández, V. Sande, F. J. Nieto, J. Carnero, A. Kovacs e T. Budai, ``MSO4SC: D3.1 Detailed Specifications for the Infrastructure, Cloud Management and MSO Portal'', \textit{Mathematical Modelling, Simulation and Optimization for Societal Challenges with Scientific Computing}, 2016.

\bibitem{docker-get-started} \textit{Get Started}. Documentación oficial de Docker (\url{https://docs.docker.com/get-started/}). Consultado o 1 de xullo do 2018.

% EXEMPLO DE ARTIGO DE REVISTA
\bibitem{UdockerDoc} J. Gomes, E. Bagnaschi, I. Campos, M. David, L. Alves, J. Martins, J. Pina, A. López-García, P. Orviz, ``\textit{Enabling rootless Linux Containers in multi-user environments: The udocker tool, Computer Physics Communications}'', ISSN 0010-4655, \url{https://doi.org/10.1016/j.cpc.2018.05.021}.

\bibitem{CloudCESGA} Infraestruturas de computación: \textit{Cloud}. Páxina do CESGA (\url{https://www.cesga.es/gl/infraestructuras/computacion/Infraestructura-cloud}). Consultado o 1 de xullo do 2018.

\bibitem{FT2CESGA} Infraestruturas de computación: Finis Terrae II. Páxina do CESGA (\url{https://www.cesga.es/gl/infraestructuras/computacion/FinisTerrae2}). Consultado o 1 de xullo do 2018.

% EXEMPLO DE ARTIGO DE REVISTA
\bibitem{singularityScientificContainers} Kurtzer, M. Gregory, V. Sochat e M. W. Bauer, ``\textit{Singularity: Scientific containers for mobility of compute}'', {\it PLoS ONE 12(5):e0177,459}, 2017. \url{https://doi.org/10.1371/journal.pone.0177459}

% EXEMPLO DE PÁXINA DA WIKIPEDIA
\bibitem{singularity-limits} \textit{Limit cpus and memory in Singularity}. Discusión Google Groups (\url{https://groups.google.com/a/lbl.gov/forum/\#!topic/singularity/5byecITBQuw}). Consultado o 1 de xullo do 2018.

% EXEMEPLO DE LIBRO
\bibitem{TFM} S. Lipke, ``\textit{Building a Secure Software Supply Chain using Docker}'', Tese de Fin de Mestrado desenvolvida na \textit{Stuttgart Media University} e no \textit{ERNW Enno Rey Netzwerke GmbH}, Heidelberg, 2017.

\bibitem{docker-networking} \textit{Networking overview}. Documentación oficial de Docker (\url{https://docs.docker.com/network/}). Consultado o 1 de xullo do 2018.

\bibitem{dockerHubExp} \textit{Overview of Docker Hub}. Documentación oficial de Docker (\url{https://docs.docker.com/docker-hub/}). Consultado o 1 de xullo do 2018.

% EXEMEPLO DE LIBRO
\bibitem{PMBOK} \textit{Project Management Institute}, \textit{A guide to the project management body of knowledge (PMBOK guide)}, Newtown Square, 2004.

% EXEMPLO DE ARTIGO DE REVISTA
\bibitem{detectingARPSpoofing} V. Ramachandran e S. Nandi, ``\textit{Detecting ARP Spoofing: An Active Technique}'', \textit{
ICISS'05 Proceedings of the First international conference on Information Systems Security}, 2005, pp. 239-250.

% EXEMPLO DE ARTIGO DE REVISTA
\bibitem{OS-level-security} E. Reshetova, J. Karhunen, T. J. Nyman e N. Asokan, ``\textit{Security of OS-level virtualization technologies}'', {\it Secure IT Systems: 19th Nordic Conference, NordSec 2014, Tromsø, Norway, October 15-17, 2014, Proceedings}, pp. 77-93.

\bibitem{SingularitySecurity} \textit{Security}. Documentación oficial de Singularity (\url{http://singularity.lbl.gov/docs-security}). Consultado o 1 de xullo do 2018.

\bibitem{SELinuxMitigates} \textit{SELinux Mitigates container Vulnerability}. \textit{Red Hat Enterprise Linux Blog} (\url{https://rhelblog.redhat.com/2017/01/13/selinux-mitigates-container-vulnerability/}). Consultado o 1 de xullo do 2018.

% EXEMPLO DE ARTIGO DE REVISTA
\bibitem{state-of-art-docker-security} G. Shobha Sahana Upadhya, J. Shetty e R. Rajeshwari, ``\textit{A State-of-Art Review of Docker Container Security Issues and Solutions}'', {\it American International Journal of Research in Science, Technology, Engineering \& Mathematics Volume
17}, 2017, pp. 33-36.

% EXEMPLO DE ARTIGO DE REVISTA
\bibitem{studySecurityDockerHub} R. Shu, X. Gu e W. Enck, ``\textit{A Study of Security Vulnerabilities on Docker Hub}'', {\it CODASPY '17 Proceedings of the Seventh ACM on Conference on Data and Application Security and Privacy}, 2017, pp. 269-280.

\bibitem{SingularityHPC} \textit{Singularity on HPC}. Documentación oficial de Singularity (\url{http://singularity.lbl.gov/docs-hpc}). Consultado o 1 de xullo do 2018.

% EXEMEPLO DE LIBRO
\bibitem{Sommerville} I. Sommerville, ``Ingeniería del software'', 5ª edición, Pearson Addison Wesley, Madrid, 2002.

\bibitem{SingularityLabNotes} \textit{Sylabs Lab notes and news}. Documentación oficial de Sylabs (\url{https://www.sylabs.io/lab-notes/}). Consultado o 1 de xullo do 2018.

\bibitem{SingularityRemoteBuild} \textit{Sylabs Remote Build Service}. Documentación oficial de Sylabs (\url{https://www.sylabs.io/2018/04/sylabs-remote-build-service/}). Consultado o 1 de xullo do 2018.

\bibitem{docker-bridge-networks} \textit{Use bridge networks}. Documentación oficial de Docker (\url{https://docs.docker.com/network/bridge/}). Consultado o 1 de xullo do 2018.

\bibitem{clair} \textit{Vulnerability Static Analysis for Containers}. Repositorio GitHub de Clair (\url{https://github.com/coreos/clair}). Consultado o 1 de xullo do 2018.

\bibitem{xestionEconomicaUSC} Xestión económica en I+D. Documentación da Universidade de Santiago de Compostela (\url{https://imaisd.usc.es/ftp/oit/documentos/591_gl.pdf}). Consultado o 1 de xullo do 2018.


\end{thebibliography}